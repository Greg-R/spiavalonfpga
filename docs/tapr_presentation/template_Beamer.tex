\documentclass{beamer}

\usepackage{default}
\usetheme{Madrid}
\usepackage{parskip}
\usepackage{hyperref}

%Information to be included in the title page:
\title{An FPGA Learning Experience: SPI Interface to Max10 FPGA}
\subtitle{ARRL and TAPR 38th Annual Digital Communications Conference}

\author{Gregory Raven KF5N}
\date{September 21 2019}
\logo{\includegraphics[height=1.1cm]{./graphics/tapr-logo.png}}
\begin{document}
	
	\frame{\titlepage}
	
	\begin{frame}
	\frametitle{Biography}
	Gregory Raven, first licensed novice WD5HUV in 1978.
	
	KF5N since 1980.
	
	35 Year Motorolan, entire career designing Two-way FM Radios.
	Mostly interested in Ham Radio experimenting and science.
	Residing in Florida, all of my antennas have been blown down by hurricanes or struck by lightning.
	
	NOT a professionally trained digital hardware engineer!  I am a very "Analog-RF" engineer.
	Eventually I decided learning digital was a good idea, and I attended the TAPR conference in Maryland in 2011.  Slowly going up the learning curve since then ...
\end{frame}

	\begin{frame}
	\frametitle{The Golden Age of Single Board Computers}
	
	My early introduction to "digital" was via Arduino, and later Beaglebone and Raspberry Pi.
	The so-called "Single Board Computer" or ``SBC''.
	
	I spent a couple of years working with SBCs.  This was greatly accelerated when I got my hands on ``Exploring Beaglebone, Tools and Techniques for Building with Embedded Linux''.  Great book!
	
	Looking back, I think TAPR St Petersburg 2016 was essentially an SBC-themed conference!
	Obviously there is a huge opportunity for SBC in Amateur Radio.
	
\end{frame}

\begin{frame}
\frametitle{Prerequisites to begin FPGA?}


\begin{itemize}
	\item Digital Hardware experience (but not that much!)
\item Work with SBCs for a while.  Finish some projects.
\item C or similar programming.  You did this working with SBCs?
\item Discretionary time to expend.  FPGA is not rocket science, but it is time consuming!
\item Optional:  Projects with some flavor of RTOS.  Helps understanding of why FPGA is great! (ESP32, FreeRTOS, \$10)
\end{itemize}


\end{frame}

\begin{frame}
\frametitle{The State of ``Hobbyist'' FPGA}

``Hobbyist'' FPGA is miniscule compared to SBC:

\begin{itemize}
	\item Dozens, maybe hundreds of SBC books.  FPGA: 4
	\item Fair representation on hackster.io and hackaday.io.
	\item ARRL bookstore: lots of Arduino, RPi, nothing on FPGA.
\end{itemize}

As usual, Amateur Radio experimenters were pioneers in new technology, having used FPGA to create early state-of-the-art ``Software Defined Radio'' transceivers.  This goes back to at least the early 2000s, and perhaps years earlier than that.  Does anyone know the first Amateur Radio application of FPGA? 

\end{frame}

\begin{frame}
\frametitle{FPGA Boards?}

What is out there to work with?

\url{https://makezine.com/comparison/boards/}

Currently, most FPGA are ``development boards'' intended for industrial or academic applications.

There is a little bit of good news!  Will return to this topic later ...

\end{frame}

\begin{frame}
\frametitle{FPGAs in the Silicon Ecosystem}

Boiling it down to fundamental capabilities:

INSERT ASIC FPGA SOC diagram here.


\end{frame}

\begin{frame}
\frametitle{What is exciting about FPGAs?}

``3-D Printer for Electronics?''

HARDWARE not software!

This gives you the power to create multiple specifically targeted digital machines which can work independently of one another.  This is the capability that SDR uses to crunch DSP math very efficiently while handling data flow simultaneously.  A list of FPGA applications from Wikipedia:

\url{https://en.wikipedia.org/wiki/Field-programmable_gate_array}

\end{frame}

\begin{frame}
\frametitle{How Do You Begin?}

For FPGA this is not as clear as SBCs.  An SBC beginner I would say to get an Arduino UNO and a good book or website and get going.  Later graduate to embedded Linux and RPi, Beaglebone, or equivalent.

So I will give an example here, not correct for everyone ...
Lectures by Bruce Land of Cornell University using PIC32:

\begin{tiny}
\url{https://www.youtube.com/watch?v=FYy6JN0vpg0&list=PLKcjQ_UFkrd4z2qoFuJ1jtVhCSuxxCTpk}
\end{tiny}

This course uses a PIC32 microcontroller board.  However, the material in the lectures is generic enough to apply to any board which can be programmed in C.
Even better, a second course covers FPGA!

\begin{scriptsize}
\url{https://www.youtube.com/playlist?list=PLKcjQ_UFkrd7UcOVMm39A6VdMbWWq-e_c}
\end{scriptsize}

and the matching website:

\url{http://people.ece.cornell.edu/land/courses/ece5760/}

\end{frame}

\begin{frame}
\frametitle{Online MOOC Style Courses?}

I tried:

\begin{itemize}
	\item Coursera
\item Udemy
\item Hackster.io
\end{itemize}

I didn't get much out of the above.  They didn't really follow through to a practical application.  Hopefully some new courses on FPGA will appear, and I'm sure they will.
So far my experience with FPGA MOOCs is underwhelming.

\end{frame}

\begin{frame}
\frametitle{Choosing a Development Board}

As mentioned earlier, the choice is limited compared to SBCs.  Most are targeted at  industrial/academic application.  Many are expensive!

The Cornell FPGA course uses an "SOC" style device, and the board cost is \$250.
For someone who is not sure about FPGA, that is too much!

Also, an ``SOC'' is too much for a beginner.  This has two Cortex A9 processors, runs its own flavor of embedded Linux, and it looks overwhelming!

The same company making the DE1-SOC also makes others:

{\tiny \url{https://www.terasic.com.tw/cgi-bin/page/archive.pl?Language=English&CategoryNo=163}}

\end{frame}

\begin{frame}
\frametitle{What I Selected}

Terasic DE10-Lite is a cost-effective Altera MAX 10 based FPGA board. The board utilizes the maximum capacity MAX 10 FPGA, which has around 50K logic elements(LEs) and on-die analog-to-digital converter (ADC). It features on-board USB-Blaster, SDRAM, accelerometer, VGA output, 2x20 GPIO expansion connector, and an Arduino UNO R3 expansion connector in a compact size. The kit provides the perfect system-level prototyping solution for industrial, automotive, consumer, and many other market applications.

The DE10-Lite kit also contains lots of reference designs and software utilities for users to easily develop their applications based on these design resources.

{\tiny
\url{https://www.terasic.com.tw/cgi-bin/page/archive.pl?Language=English&CategoryNo=234&No=1021}}

\end{frame}

\begin{frame}
\frametitle{What I Selected}

\begin{figure}[h]
	\centering
	\includegraphics[width=0.9\textwidth]{graphics/de10-lite.jpg}
	\centering\bfseries
	\caption{Terasic DE10-Lite \$85}
\end{figure}

\end{frame}

\begin{frame}
\frametitle{Wait!!! Don't Put it in the Cart and Purchase}

This is not a moment for ``instant gratification''!  You have some work to do first.

Before you take the plunge, you MUST install the tools required to make this thing work!
I chose an Intel (formerly Altera) based board (MAX10), so this means installing ``Quartus''.  Xilinx devices will require ``Vivado''.

But to do this, you should get a ``free'' account at the Intel site:

\url{https://www.intel.com/content/www/us/en/products/programmable.html}

Go here for ``Quartus Prime Lite'':

\url{http://fpgasoftware.intel.com/?edition=lite}

Make sure you can install and run this large beast!

\end{frame}

\begin{frame}
\frametitle{Development Platforms}

I am a huge fan of Linux and I almost always use a distribution of Ubuntu.
Quartus has a version for Windows, however, I have not tested it.

Regarding Ubuntu, I don't recommend running Quartus in Ubuntu directly.  The simulation tool, Vsim, will not run.

Instead, use a Virtual Box to create a ``virtual machine''.  I experimented with Centos 6 and 7.  Centos 6 was the easiest to configure.  Even so, there were a couple of tricks to get everything working smoothly.

If you go the route of a virtual machine, you will need to work out permissions with USB devices, as the DE10-Lite uses the so-called ``USB Blaster'' mechanism for programming.

Quartus is a significant user of desktop real estate.  I don't think anything less than full HD (1920x1080) is going to be satisfactory.  I found FHD to be marginal, and I bought a new 27 inch ``QHD'' display, which is 2560x1440 pixels.  Much better!

\end{frame}

\begin{frame}
\frametitle{Project Inspiration:  JTAG Interface to DE10-Lite}

I had abandoned the SOC with embedded Linux as too complex, however, I still wanted a means of ``talking'' to the FPGA, preferably from my Ubuntu box.

After a bunch of searching, I found this:

\url{https://github.com/hildebrandmw/de10lite-hdl}

The interesting feature of this project is the usage of the USB to load data (animated GIF image file) to
the FPGA. So this is a data pipeline from a Linux desktop to the FPGA which is built into the board! This
met my requirement that I be able to control the FPGA remotely from a desktop (or SBC) computer.
The interface is via “JTAG”, which is typically used as a debugging interface. It is not specifically intended
for mass data transfer, but in this case it was pressed into service.
The interface is a bit clunky to use. It requires a “TCL Server”, and a running instance of the Quartus
development tool!

\end{frame}

\begin{frame}
\frametitle{Project Inspiration:  JTAG Interface to DE10-Lite}

\begin{figure}[h]
	\centering
	\includegraphics[width=1.0\textwidth]{graphics/play_gif.png}
	\centering\bfseries
	\caption{Play\_gif JTAG Interface System Diagram}
\end{figure}

\end{frame}

\begin{frame}
\frametitle{IP and Platform Designer}

FPGA Jargon:  ``Intellectual Property'' or ``IP''

First, a little bit of FPGA jargon.  ``Intellectual Property'' (IP) in the context of semiconductor devices is a block of circuitry which has been heavily engineered and refined to perform some particular function.  It could be patented or otherwise protected from duplication by competitors.
Due to the way integrated circuits are manufactured, blocks of ''IP'' can be added to the silicon and be expected to perform to the IP owner's specifications.  Typically IP can be included as part of a design kit, or it can be paid for with a license fee.

IP is good because it can reduce engineering design effort, improve performance, and enhance quality.  The trade-off is license fee cost, and you don't necessarily get exactly what you want.

In our case, we are given a whole bunch of IP for free that we can experiment with!  This is bundled into ``Platform Designer'' which is a tool-within-a-tool in the Quartus design suite.

\end{frame}

\begin{frame}
\frametitle{IP and Platform Designer}

\begin{figure}[h]
	\centering
	\includegraphics[width=1.0\textwidth]{graphics/platform_designer.pdf}
	\centering\bfseries
	\caption{Play\_gif JTAG Interface System Diagram}
\end{figure}

\end{frame}

\end{document}
