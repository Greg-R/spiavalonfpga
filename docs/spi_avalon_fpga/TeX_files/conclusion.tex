\section{Simulation}

The FPGA developer should develop skills in FPGA simulation.  The original ``blinky LED'' project I attempted prior to the SPI bus project required me to do that.
I had a minor, but persistent bug which brought my work to a halt.  Trial and error with the board and re-doing the FPGA got me nowhere fast.  After I had a simple simulation running, the problem was quickly resolved.
Fortunately the Quartus tool does most of the work to set up the simulation.  A small bit of hacking of ``do'' (TCL) files is required.

I recommend using the resources on the Intel web site to explore the basics of setting up and running simulations via Quartus.

\section{Conclusion}

I was able to find an FPGA development board and ``starter project'' that met my requirements for a beginning in FPGA development.
A bare minimum of Verilog modifications were required to make the project function correctly.

Most of the FPGA ``design'' was done using the system-level ``Platform Designer'' tool.

I was able to add memory-mapped read of the SDRAM and FPGA registers via analysis of the Avalon bus structure.  The Julia programming
language was used to create and decode Avalon bus transactions.  Direct access of the SPI device C shared library from Julia code was used.

The next stage of the project will be a practical ham radio application.  I have an antenna rotator which I want to control via wireless.
I think it will be possible to create a real-time state machine on the FPGA which will handle the motor drive along with a Hall sensor
and counter for positional feedback.  Driving the FPGA from the SPI port of an SBC, along with a WIFI connection, should allow the entire
control unit to be wireless.  Solar powered?  Maybe.

