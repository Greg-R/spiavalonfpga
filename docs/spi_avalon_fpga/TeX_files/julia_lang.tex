\section{SPI Driver in Julia Programming Language}

The github project from which mine was derived uses a programming language called ``Julia'':

\url{https://julialang.org/}

This was a language I had heard about, but I've never tried it.
To duplicate the original project, I had to install the language and run the program.
The program does the processing of the GIF image, and then sends the data to the FPGA
via the JTAG server.

It was simple to install the server and run the Julia program.  It all worked first time!

Rather than reinventing the wheel, I decided to use the image processing portion of the Julia program.
However, how to interface to the FTDI SPI device?

The FTDI device is supported with a C shared library.  This library has the initialization, read, write, and shutdown
command necessary to work with the device.

Fortuitously, the Julia language includes the capability to call C library functions in a very direct way!
I was skeptical at first, but I quickly had the SPI device's initialization function running and returning with no error.
The other required functions were quickly added.  I now had full control of the SPI bus from the command line!

When I say ``command line'', in the case of Julia I am referring to the ``Read Eval Print Loop'', called the REPL.  This functionality is similar to Python, and is my favorite way to develop code.  I also used the ``Atom'' IDE, which has a plug-in for the Julia language.
It is a new language, and has a few quirks like all of them do, but so far I am very impressed!

At first, Julia was also used to access a shared library which was taken from an Altera demo project of the SPI-Avalon bus master.
This library was responsible for reading and writing ``Avalon Packets''.  This is the protocol used by the Avalon bus.  I was able to successfully translate the Altera library to Julia.  This is working well and is able to read and write to the SPI and thus the Avalon bus.

The Julia code is located in the ``julia'' directory of the github repository linked in the introduction.