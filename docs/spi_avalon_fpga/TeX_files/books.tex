\section{FPGA and Verilog Books}

Notes on books which might be useful to an FPGA beginner.

\subsection{Hobbyist FPGA Books}

"Designing Video Game Hardware in Verilog"
Steven Hugg, first printing 2018

A very practical introduction to digital hardware using the early history of video games as a means of illustrating the technology.
The book includes significant introduction to Verilog and its relationship to the physical circuitry.

However, it is not primarily an FPGA book!  The reader is encouraged to use a web-based Verilog simulator:

\url{http://8bitworkshop.com}

There is an example FPGA which uses the iCE40HX-1K iCEstick.
This is a low-cost device (\$40 on Amazon).  Development can proceed
with the official Lattice tool chain, or with an open-source system
known as ``IceStorm'':

\url{http://www.clifford.at/icestorm}

``Programming FPGAs, Getting Started in Verilog''
Simon Monk, 2017 McGraw-Hill Education

Good coverage of beginning FPGA using the boards Elbert 2, Mojo, and Papilio.

\begin{footnotesize}
\url{https://www.amazon.com/Programming-FPGAs-Getting-Started-Verilog/dp/125964376X/ref=sr_1_1?keywords=fpga+monk&qid=1564448368&s=books&sr=1-1}
\end{footnotesize}

\subsection{Verilog}

You will need to go up the learning curve on Verilog (or VHDL).
Here are a couple of inexpensive books (< \$20) which will get you going:

``Designing Digital Systems with SystemVerilog''
Brent E. Nelson, Brigham Young University

\begin{footnotesize}
\url{https://www.amazon.com/gp/product/1075968437/ref=ppx_yo_dt_b_asin_title_o00_s03?ie=UTF8&psc=1}
\end{footnotesize}

``Exploring Digital Logic''
George Self.

\begin{footnotesize}
\url{http://www.lulu.com/shop/george-self/exploring-digital-logic/paperback/product-22747579.html}
\end{footnotesize}


